%2345678901234567890123456789012345678901234567890123456789012345678901234567890
%        1         2         3         4         5         6         7         8

%%%%%%%%%%%%%%%%%%%%%%%%%%%%%%%%%%%%%%%%%%%%%%%%%%%%%%%%%%%%%%%%%%%%%%%%%%%%%%%%
% PACKAGES                                                                     %
%%%%%%%%%%%%%%%%%%%%%%%%%%%%%%%%%%%%%%%%%%%%%%%%%%%%%%%%%%%%%%%%%%%%%%%%%%%%%%%%
\documentclass[a4paper, 10pt, twoside]{amsart}
%\documentclass[a4paper, 12pt, twoside]{article}

%------------------------------------------------------------------------------%
% PAGE LAYOUT                                                                  %
%------------------------------------------------------------------------------%
\usepackage[top=2cm, bottom=2cm, left=2cm, right=2cm]{geometry}
%\usepackage{fancyhdr} % More control on header and footer

%------------------------------------------------------------------------------%
% TEXT FORMATTING                                                              %
%------------------------------------------------------------------------------%
%\usepackage{setspace} % Sets the space between lines
%\usepackage{anyfontsize} % Fonts can be scaled to any size.

%------------------------------------------------------------------------------%
% IMAGES                                                                       %
%------------------------------------------------------------------------------%
\usepackage[pdftex]{graphicx}
\pdfsuppresswarningpagegroup=1 % A warning issued when several PDF images are
% imported in the same page. Mostly harmless, can be almost always supressed.
\usepackage[pstarrows]{pict2e} % More commands for the picture environment
\usepackage{tikz} % Best way of drawing pictures
\usetikzlibrary{shapes, arrows, arrows.meta, shapes.misc} % Some useful tikz libraries

%------------------------------------------------------------------------------%
% TABLES                                                                       %
%------------------------------------------------------------------------------%
\usepackage{array} % More options for table formatting
%\usepackage{multirow} % Cells that occupy multiple rows
%\usepackage{longtable} % Tables that occupy multiple pages

%------------------------------------------------------------------------------%
% AMS PACKAGES                                                                 %
%------------------------------------------------------------------------------%
\usepackage{amsmath} % Basic AMS-LaTeX package, with most common features.
                     % Automatically loads:
										 %     amstext: for \text
										 %     amsbsy: for \boldsymbol and \pmb
										 %     amsopn: for \DeclareMathOperator
										 % Automatically loaded on AMS document classes.
%\usepackage{amsfonts} % Mathematical symbols, including \mathbb and \mathfrak.
%\usepackage{amscd} % Package for commutative diagrams
\usepackage{amsthm} % Extended functionalities for \newtheorem.
                   % Automatically loaded on AMS document classes.
\usepackage{amssymb} % Extended mathematical symbols.
                     % May exceed LaTeX symbol capacity.
                     % Automatically loads:
										 %     amsfonts

%------------------------------------------------------------------------------%
% MATHEMATICAL SYMBOLS                                                         %
%------------------------------------------------------------------------------%
\usepackage{mathtools} % More symbols (eg. \xrightharpoon, \coloneqq)
\usepackage[integrals]{wasysym} % Changes the style of the integrals
\usepackage[nice]{nicefrac} % Better fractions
\usepackage{stmaryrd} % Symbols (eg. \llbracket and \rrbracket for intervals of
%                       integers)

%------------------------------------------------------------------------------%
% MATHEMATICAL FONTS                                                           %
%------------------------------------------------------------------------------%
%\usepackage{mathbbol} % Almost every symbol with \mathbb
\usepackage{bbm} % More symbols with \mathbb. Use \mathbbm
%\usepackage{calrsfs} % Changes the style of \mathcal
\usepackage[mathcal]{euscript} % Changes the style of \mathcal

%------------------------------------------------------------------------------%
% FONT ENCODING                                                                %
%------------------------------------------------------------------------------%
\usepackage[utf8]{inputenc} % Encoding of the .tex file.
\usepackage[T1]{fontenc} % T1 fonts, necessary to treat accentuated characters
%                          as a single block.

%------------------------------------------------------------------------------%
% LANGUAGES                                                                    %
%------------------------------------------------------------------------------%
\usepackage[brazil, french, english]{babel} % Selects the language of the
%                              document, defining the names of the sections,
%                              table of contents, bibliography, etc. The default
%                              language for a multi-language document is the
%                              last one.
%\frenchsetup{StandardLists=true} % Necessary when writing documents in French
%                                    with the package enumitem to avoid
%                                    conflicts with the environment itemize
%\NoAutoSpaceBeforeFDP % Suppresses the automatic space before :;?!

%------------------------------------------------------------------------------%
% BIBLIOGRAPHY                                                                 %
%------------------------------------------------------------------------------%
%\usepackage{babelbib} % Used to define the language of each bibliography entry.
%                       Use [fixlanguage] to use the same language for every
%                       entry and \selectbiblanguage{} to define it. An adapted
%                       style must be used (such as babplain).
\usepackage{cite} % Organizes the entries inside a single \cite.

%------------------------------------------------------------------------------%
% FONTS                                                                        %
%------------------------------------------------------------------------------%
% Computer Modern (default font)                                               %
% - - - - - - - - - - - - - - - - - - - - - - - - - - - - - - - - - - - - - - -%
\usepackage{ae} % To be used with Computer Modern when generating PDFs to
%                  correct some errors.

% Computer Modern Bright (sans serif)                                          %
% - - - - - - - - - - - - - - - - - - - - - - - - - - - - - - - - - - - - - - -%
%\usepackage{cmbright}

% Times New Roman                                                              %
% - - - - - - - - - - - - - - - - - - - - - - - - - - - - - - - - - - - - - - -%
%\usepackage{mathptmx} % Text and math mode
%\usepackage{times} % Only text, doesn't change math mode
%\usepackage{newtxtext, newtxmath} % Contains bold greek letters

% Arial                                                                        %
% - - - - - - - - - - - - - - - - - - - - - - - - - - - - - - - - - - - - - - -%
%\usepackage[scaled]{uarial} % Arial as the default sans serif font

% Palatino                                                                     %
% - - - - - - - - - - - - - - - - - - - - - - - - - - - - - - - - - - - - - - -%
%\usepackage{mathpazo} % Text and math mode
%\usepackage{palatino} % Only text, doesn't change math mode

% Concrete                                                                     %
% - - - - - - - - - - - - - - - - - - - - - - - - - - - - - - - - - - - - - - -%
%\usepackage{ccfonts} % Text: Concrete; Math mode: Concrete Math
%\usepackage{ccfonts, eulervm} % Text: Concrete; Math mode: Euler

% Iwona                                                                        %
% - - - - - - - - - - - - - - - - - - - - - - - - - - - - - - - - - - - - - - -%
%\usepackage[math]{iwona} % Text and math mode: Iwona

% Kurier                                                                       %
% - - - - - - - - - - - - - - - - - - - - - - - - - - - - - - - - - - - - - - -%
%\usepackage[math]{kurier} % Texto and math mode: Kurier

% Antykwa Póltawskiego                                                         %
% - - - - - - - - - - - - - - - - - - - - - - - - - - - - - - - - - - - - - - -%
%\usepackage{antpolt} % Text: Antykwa Póltawskiego; Math mode: none
                     % Use fontenc = QX or OT4

% Utopia                                                                       %
% - - - - - - - - - - - - - - - - - - - - - - - - - - - - - - - - - - - - - - -%
%\usepackage{fourier} % Text: Utopia; Math mode: Fourier

% KP Serif                                                                     %
% - - - - - - - - - - - - - - - - - - - - - - - - - - - - - - - - - - - - - - -%
%\usepackage{kpfonts}

%------------------------------------------------------------------------------%
% OTHER PACKAGES                                                               %
%------------------------------------------------------------------------------%
%\usepackage[section]{algorithm} % To typeset algorithms
%\usepackage{appendix} % Creates subappendices (appendice sections in the end
%                        of a chapter).
\usepackage{calc} % Computations with lengths and counters.
\usepackage[inline]{enumitem} % Better lists, including references to items of
%                               lists
%\usepackage{icomma} % Uses comma as the decimal separator
%\usepackage{lipsum} % Lorem ipsum text
%\usepackage{pdflscape} % Landscape pages
%\usepackage{pdfpages} % Allows the inclusion of PDF files
%\usepackage{randtext} % Permutes the letters of a sentence stored on the PDF
                       % file (useful for e-mail addresses in documents to be
					   % made available on-line).
%\usepackage{refcheck} % Checks the references looking for unused labels
%                      % Look for RefCheck in the .log file.
\usepackage[notref]{showkeys} % Shows the names of the labels
\renewcommand*\showkeyslabelformat[1]{\normalfont\tiny\ttfamily#1}
% Format for showing label names
\usepackage{url} % Use \url{} to declare a URL
%\usepackage{xcolor} % Automatically loaded by tikz.

%------------------------------------------------------------------------------%
% HYPERLINKS                                                                   %
%------------------------------------------------------------------------------%
\usepackage[linktocpage, colorlinks, citecolor=blue, urlcolor=violet, bookmarks, bookmarksnumbered, pdfstartview={XYZ null null 1.00}]{hyperref}

%%%%%%%%%%%%%%%%%%%%%%%%%%%%%%%%%%%%%%%%%%%%%%%%%%%%%%%%%%%%%%%%%%%%%%%%%%%%%%%%
% CONFIGURATIONS                                                               %
%%%%%%%%%%%%%%%%%%%%%%%%%%%%%%%%%%%%%%%%%%%%%%%%%%%%%%%%%%%%%%%%%%%%%%%%%%%%%%%%

%------------------------------------------------------------------------------%
% TEXT FORMATTING                                                              %
%------------------------------------------------------------------------------%
%\onehalfspacing % 1 1/2 spacing (defined in the package setspace)

%------------------------------------------------------------------------------%
% DEFINITION OF MATH ENVIRONMENTS                                              %
%------------------------------------------------------------------------------%
\iflanguage{french}{
  \newcommand{\theoname}{Théorème}
  \newcommand{\lemmname}{Lemme}
  \newcommand{\coroname}{Corolaire}
  \newcommand{\propname}{Proposition}
  \newcommand{\definame}{Définition}
  \newcommand{\hyponame}{Hypothèses}
  \newcommand{\remkname}{Remarque}
  \newcommand{\explname}{Exemple}
  \newcommand{\soluname}{Solution}
}{
\iflanguage{portuguese}{
  \newcommand{\theoname}{Teorema}
  \newcommand{\lemmname}{Lema}
  \newcommand{\coroname}{Corolário}
  \newcommand{\propname}{Proposição}
  \newcommand{\definame}{Definição}
  \newcommand{\hyponame}{Hipóteses}
  \newcommand{\remkname}{Observação}
  \newcommand{\explname}{Exemplo}
  \newcommand{\soluname}{Solução}
}{ % Default: english
  \newcommand{\theoname}{Theorem}
  \newcommand{\lemmname}{Lemma}
  \newcommand{\coroname}{Corollary}
  \newcommand{\propname}{Proposition}
  \newcommand{\definame}{Definition}
  \newcommand{\hyponame}{Hypotheses}
  \newcommand{\remkname}{Remark}
  \newcommand{\explname}{Example}
  \newcommand{\soluname}{Solution}
}}

\theoremstyle{plain}
\newtheorem{theorem}{\theoname}[section]
\newtheorem{lemma}[theorem]{\lemmname}
\newtheorem{corollary}[theorem]{\coroname}
\newtheorem{proposition}[theorem]{\propname}
\theoremstyle{definition}
\newtheorem{definition}[theorem]{\definame}
\newtheorem{hypothesis}[theorem]{\hyponame}
\newtheorem{remark}[theorem]{\remkname}
\newtheorem{example}[theorem]{\explname}

%------------------------------------------------------------------------------%
% DEFINITION OF MATH OPERATORS                                                 %
%------------------------------------------------------------------------------%
\DeclareMathOperator{\Lip}{Lip} % Lipschitz continuous functions
\DeclareMathOperator{\sign}{sign} % Signal
\DeclareMathOperator{\diam}{diam} % Diameter
\DeclareMathOperator{\Ker}{Ker} % Kernel
\DeclareMathOperator{\Real}{Re} % Real part
%\DeclareMathOperator{\Tr}{Tr} % Trace
%\DeclareMathOperator{\spr}{\rho} % Spectral radius
\DeclareMathOperator{\conv}{co} % Convex hull
\DeclareMathOperator{\diverg}{div} % Divergence
\DeclareMathOperator{\rank}{rk} % Rank
\DeclareMathOperator{\range}{Ran} % Range
\DeclareMathOperator{\diag}{diag} % Diagonal matrix
\DeclareMathOperator{\id}{Id} % Identity matrix
\DeclareMathOperator{\meas}{meas} % Measure
\DeclareMathOperator{\Span}{Span} % Vector span
\DeclareMathOperator{\Vol}{Vol} % Volume
\DeclareMathOperator{\supp}{supp} % Support
\DeclareMathOperator{\Proj}{Pr} % Projection
\DeclareMathOperator*{\esssup}{ess\,sup} % Essential supremum
\DeclareMathOperator{\diff}{d\!} % Differential
\DeclareMathOperator{\transp}{T} % Transpose

\DeclarePairedDelimiter{\norm}{\lVert}{\rVert} % Norm
\DeclarePairedDelimiter{\abs}{\lvert}{\rvert} % Absolute value
\DeclarePairedDelimiter{\floor}{\lfloor}{\rfloor} % Floor
\DeclarePairedDelimiter{\ceil}{\lceil}{\rceil} % Ceiling
\DeclarePairedDelimiter{\pfrac}{\{}{\}} % Fractionary part
\DeclarePairedDelimiter{\average}{\langle}{\rangle} % Average
\DeclarePairedDelimiter{\scalprod}{\langle}{\rangle} % Scalar product

%\newcommand{\ind}{\mathbbmss 1} % Indicator. Uses the package bbm
\newcommand{\suchthat}{\ifnum\currentgrouptype=16 \mathrel{}\middle|\mathrel{}\else\mid\fi}

\newcommand{\bracket}[2]{\left\langle#1\middle|#2\right\rangle} % Bracket

%------------------------------------------------------------------------------%
% FLOATS                                                                       %
%------------------------------------------------------------------------------%
% Maximal percentage of the page occupied by floats
\renewcommand\floatpagefraction{.9}
\renewcommand\topfraction{.9}
\renewcommand\bottomfraction{.9}
\renewcommand\textfraction{.1}
% Maximal number of floats per page
\setcounter{totalnumber}{50}
\setcounter{topnumber}{50}
\setcounter{bottomnumber}{50}

%------------------------------------------------------------------------------%
% DOCUMENT STRUCTURE                                                           %
%------------------------------------------------------------------------------%
\setcounter{secnumdepth}{6} % Part, chapter, section, subsection, subsubsection
%                             and paragraph have numbers.
\setcounter{tocdepth}{6} % Part, chapter, section, subsection, subsubsection and
%                          paragraph in the table of contents.

%------------------------------------------------------------------------------%
% NUMBERS OF FIGURES, TABLES AND EQUATIONS                                     %
%------------------------------------------------------------------------------%
\numberwithin{table}{section}
%\numberwithin{table}{subsection}
\numberwithin{figure}{section}
%\numberwithin{figure}{subsection}
\numberwithin{equation}{section}
%\numberwithin{equation}{subsection}
%\numberwithin{theo}{chapter}
%\numberwithin{theo}{subsection}

%%%%%%%%%%%%%%%%%%%%%%%%%%%%%%%%%%%%%%%%%%%%%%%%%%%%%%%%%%%%%%%%%%%%%%%%%%%%%%%%
%%%%%%%%%%%%%%%%%%%%%%%%%%%%%%%%%%%%%%%%%%%%%%%%%%%%%%%%%%%%%%%%%%%%%%%%%%%%%%%%
%% DOCUMENT                                                                   %%
%%%%%%%%%%%%%%%%%%%%%%%%%%%%%%%%%%%%%%%%%%%%%%%%%%%%%%%%%%%%%%%%%%%%%%%%%%%%%%%%
%%%%%%%%%%%%%%%%%%%%%%%%%%%%%%%%%%%%%%%%%%%%%%%%%%%%%%%%%%%%%%%%%%%%%%%%%%%%%%%%
\begin{document}

%------------------------------------------------------------------------------%
% DOCUMENT COMMANDS                                                            %
%------------------------------------------------------------------------------%
\setlength{\parskip}{1pt plus 1pt minus 1pt} % Useful to correct vertical 
%                                              spacing issues with AMS classes

\pagestyle{plain}

% Configure some enumitem styles
\setlist[enumerate, 1]{label={\textnormal{(\alph*)}}, ref={(\alph*)}, leftmargin=0pt, itemindent=*}
\setlist[enumerate, 2]{label={\textnormal{(\roman*)}}, ref={(\roman*)}}
\setlist[description, 1]{leftmargin=0pt, itemindent=*}
\setlist[itemize, 1]{label={\textbullet}, leftmargin=0pt, itemindent=*}

\begin{center}
\LARGE 3I025 --- Projet

\Large Ariana CARNIELLI et David HERZOG
\end{center}

\bigskip

On a deux versions du joueur, une première que l’on appelle A basée sur un arbre de décision simple avec des paramètres optimisés par l’algorithme génétique et une seconde, appelée B, basée sur une arbre de décision plus complexe.

L’arbre de décision de A a trois comportements de base : explorer l’espace en évitant murs et autres robots (EXPLORE), suivre un robot ennemi (FOLLOW) et essayer de dévier d’un robot ennemi qui nous suit (DEVIATE). Dans les comportements EXPLORE et FOLLOW, on regarde aussi si le robot est bloqué et fait une manœuvre pour le débloquer si nécessaire. L’arbre A est donné dans la figure suivante :

L’optimisation par l’algorithme génétique utilise 44 paramètres, correspondant à deux types de robots, chacun à 22 paramètres. Ces paramètres comprennent 8 coefficients multipliant le vecteur des capteurs pour chacun des comportements EXPLORE et FOLLOW, 2 coefficients déterminant le comportement DEVIATE, un coefficient pour la vitesse de FOLLOW et un autre pour l’angle à partir du quel on suit un robot, un coefficient pour le bruit dans l’exploration et un dernier coefficient déterminant combien de pas sur la même position déterminent un état bloqué.

\begin{center}
\resizebox{!}{0.88\height}{
\begin{tikzpicture}
\node[draw, shape = rectangle, text width=3cm, minimum height=1.6cm, thick, align=center] (A) at (0, 0) {Y a-t-il un ennemi derrière ?};
\node[draw, shape = rounded rectangle, text width=3cm, minimum height=0.8cm, thick, align=center] (B) at (-2, -3) {DEVIATE};
\node[draw, shape = rectangle, text width=3cm, minimum height=1.6cm, thick, align=center] (C) at (2, -3) {Y a-t-il un ennemi devant ?};
\node[draw, shape = rounded rectangle, text width=3cm, minimum height=0.8cm, thick, align=center] (D) at (0, -6) {EXPLORE};
\node[draw, shape = rectangle, text width=3cm, minimum height=1.6cm, thick, align=center] (E) at (4, -6) {Y a-t-il un ami devant ?};
\node[draw, shape = rectangle, text width=3cm, minimum height=1.6cm, thick, align=center] (F) at (2, -9) {Id plus grand que le mien ?};
\node[draw, shape = rectangle, text width=3cm, minimum height=1.6cm, thick, align=center] (G) at (6, -9) {Est-on face à l'ennemi ?};
\node[draw, shape = rounded rectangle, text width=3cm, minimum height=0.8cm, thick, align=center] (H) at (4, -12) {EXPLORE};
\node[draw, shape = rounded rectangle, text width=3cm, minimum height=0.8cm, thick, align=center] (I) at (8, -12) {FOLLOW};

\draw[thick, -Stealth] (A) -- node[midway, left] {oui} (B);
\draw[thick, -Stealth] (A) -- node[midway, right] {non} (C);
\draw[thick, -Stealth] (C) -- node[midway, left] {oui} (D);
\draw[thick, -Stealth] (C) -- node[midway, right] {non} (E);
\draw[thick, -Stealth] (E) -- node[midway, left] {oui} (F);
\draw[thick, -Stealth] (E) to[out=-45, in=90] node[midway, right] {non} (G);
\draw[thick, -Stealth] (F) to[out=-120, in=180] node[midway, left] {oui} (H);
\draw[thick, -Stealth] (F) to[out=-60, in=-90] node[midway, below] {non} (4, -10) -- (4, -8) to[out=90, in=135] (G);
\draw[thick, -Stealth] (G) -- node[midway, left] {oui} (H);
\draw[thick, -Stealth] (G) -- node[midway, right] {non} (I);
\end{tikzpicture}
}
\end{center}

L’arbre de décision B se décompose en trois parties, I, II et III, et contient les comportements de base suivants~:
\begin{itemize}[itemindent=*]
\item GO\_FOWARD : aller tout droite
\item AVOID\_WALLS : éviter des murs et ignorer des robots
\item FOLLOW\_ENNEMY : suivre un robot
\item AVOID\_MATES : éviter des robots amis
\item EXPLORE : éviter des robots et des murs
\item ALONG\_WALLS : longer un mur
\item STOP : s’arrêter
\item TURN\_RIGHT : tourner à droite
\item Arrière : marche arrière
\end{itemize}

La partie I détermine si le robot se déplacera en longeant les murs, ce qui n’est fait qu’au maximum par un robot (celui d’id 2). La partie II détermine si le robot doit suivre un ennemi, ce qui n’est fait qu’au maximum par deux robots (ids 0 et 2). Finalement, la partie III essaie de se débarrasser des ennemis qui nous suivent et éviter les situations de blocage. Des derniers tests permettent de donner des ajustements fins au comportement du joueur suivant les murs, en utilisant la stratégie TURN\_RIGHT.

\noindent\begin{minipage}[t]{0.5\textwidth}
\resizebox{\textwidth}{!}{
\begin{tikzpicture}
\node[draw, rectangle, dashed, align=center] at (-4, 0) {Partie I};
\node[draw, shape = rectangle, text width=3cm, minimum height=1.6cm, thick, align=center] (A) at (0, 0) {\texttt{id == 2} ?};
\node[draw, shape = rectangle, text width=3cm, minimum height=1.6cm, thick, align=center] (B) at (-2, -3) {Ami devant et pas d'ennemi devant ?};
\node[draw, shape = rounded rectangle, text width=3cm, minimum height=0.8cm, thick, align=center] (C) at (2, -3) {EXPLORE};
\node[draw, shape = rounded rectangle, text width=3cm, minimum height=0.8cm, thick, align=center] (D) at (-4, -6) {AVOID\_MATES};
\node[draw, shape = rectangle, text width=3cm, minimum height=1.6cm, thick, align=center] (E) at (0, -6) {Fin de partie ?};
\node[draw, shape = rectangle, text width=3cm, minimum height=1.6cm, thick, align=center] (F) at (-2, -9) {Début de partie ?};
\node[draw, shape = rounded rectangle, text width=3cm, minimum height=0.8cm, thick, align=center] (G) at (2, -9) {ALONG\_WALLS};
\node[draw, shape = rectangle, text width=3cm, minimum height=1.6cm, thick, align=center] (H) at (0, -12) {Mur qu'au capteur 6 ?};
\node[draw, shape = rounded rectangle, text width=3cm, minimum height=0.8cm, thick, align=center] (I) at (-4, -12) {EXPLORE};
\node[draw, shape = rounded rectangle, text width=3cm, minimum height=0.8cm, thick, align=center] (J) at (-2, -15) {GO\_FORWARD};
\node[draw, shape = rounded rectangle, text width=3cm, minimum height=0.8cm, thick, align=center] (K) at (2, -15) {ALONG\_WALLS};
\draw[thick, -Stealth] (A) -- node[midway, left] {oui} (B);
\draw[thick, -Stealth] (A) -- node[midway, right] {non} (C);
\draw[thick, -Stealth] (B) -- node[midway, left] {oui} (D);
\draw[thick, -Stealth] (B) -- node[midway, right] {non} (E);
\draw[thick, -Stealth] (E) -- node[midway, left] {non} (F);
\draw[thick, -Stealth] (E) -- node[midway, right] {oui} (G);
\draw[thick, -Stealth] (F) -- node[midway, right] {oui} (H);
\draw[thick, -Stealth] (F) -- node[midway, left] {non} (I);
\draw[thick, -Stealth] (H) -- node[midway, left] {oui} (J);
\draw[thick, -Stealth] (H) -- node[midway, right] {non} (K);
\end{tikzpicture}
}
\end{minipage}\begin{minipage}[t]{0.5\textwidth}
\resizebox{\textwidth}{!}{
\begin{tikzpicture}
\node[draw, rectangle, dashed, align=center] at (4, 0) {Partie II};
\node[draw, shape = rectangle, text width=3cm, minimum height=1.6cm, thick, align=center] (A) at (0, 0) {\texttt{id == 0} ou \texttt{id == 2} ?};
\node[draw, shape = rectangle, text width=3cm, minimum height=1.6cm, thick, align=center] (B) at (2, -3) {Y a-t-il un ennemi devant ?};
\node[draw, shape = rounded rectangle, text width=3cm, minimum height=0.8cm, thick, align=center] (C) at (-2, -3) {EXPLORE};
\node[draw, shape = rectangle, text width=3cm, minimum height=1.6cm, thick, align=center] (D) at (4, -6) {Y a-t-il un ami devant ?};
\node[draw, shape = rounded rectangle, text width=3cm, minimum height=0.8cm, thick, align=center] (E) at (0, -6) {Garder le comportement de la partie I};
\node[draw, shape = rectangle, text width=3cm, minimum height=1.6cm, thick, align=center] (F) at (4, -9) {Est-on face à l'ennemi ?};
\node[draw, shape = rounded rectangle, text width=3cm, minimum height=0.8cm, thick, align=center] (G) at (4, -12) {AVOID\_WALLS};
\node[draw, shape = rounded rectangle, text width=3cm, minimum height=0.8cm, thick, align=center] (H) at (0, -12) {FOLLOW\_ENEMY};
\draw[thick, -Stealth] (A) -- node[midway, right] {oui} (B);
\draw[thick, -Stealth] (A) -- node[midway, left] {non} (C);
\draw[thick, -Stealth] (B) -- node[midway, right] {oui} (D);
\draw[thick, -Stealth] (B) -- node[midway, left] {non} (E);
\draw[thick, -Stealth] (D) -- node[midway, right] {non} (F);
\draw[thick, -Stealth] (D) to[out=-120, in=-60] node[midway, below] {oui} (E);
\draw[thick, -Stealth] (F) -- node[midway, right] {oui} (G);
\draw[thick, -Stealth] (F) -- node[midway, left] {non} (H);
\end{tikzpicture}
}
\end{minipage}

\vfill

\begin{center}
\resizebox{!}{0.9\height}{
\begin{tikzpicture}
\node[draw, rectangle, dashed, align=center] at (4, 0) {Partie III};
\node[draw, shape = rectangle, text width=3cm, minimum height=1.6cm, thick, align=center] (A) at (0, 0) {Y a-t-il un ennemi derrière ?};
\node[draw, shape = rounded rectangle, text width=3cm, minimum height=0.8cm, thick, align=center] (B) at (-2, -3) {STOP};
\node[draw, shape = rectangle, text width=3cm, minimum height=1.6cm, thick, align=center] (C) at (2, -3) {Comportement de la partie II == AVOID\_WALLS ?};
\node[draw, shape = rounded rectangle, text width=3cm, minimum height=0.8cm, thick, align=center] (D) at (0, -6) {Garder le comportement de la partie II};
\node[draw, shape = rectangle, text width=3cm, minimum height=1.6cm, thick, align=center] (E) at (4, -6) {Suis-je bloqué ?};
\node[draw, shape = rounded rectangle, text width=3cm, minimum height=0.8cm, thick, align=center] (F) at (4, -9) {Arrière};
\draw[thick, -Stealth] (A) -- node[midway, left] {oui} (B);
\draw[thick, -Stealth] (A) -- node[midway, right] {non} (C);
\draw[thick, -Stealth] (C) -- node[midway, left] {oui} (D);
\draw[thick, -Stealth] (C) -- node[midway, right] {non} (E);
\draw[thick, -Stealth] (E) -- node[midway, right] {oui} (F);
\draw[thick, -Stealth] (E) to[out=-120, in=-60] node[midway, below] {non} (D);
\end{tikzpicture}
}
\end{center}

\end{document}